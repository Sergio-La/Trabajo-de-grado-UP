% this file is called up by thesis.tex
% content in this file will be fed into the main document
%----------------------- introduction file header -----------------------
%%%%%%%%%%%%%%%%%%%%%%%%%%%%%%%%%%%%%%%%%%%%%%%%%%%%%%%%%%%%%%%%%%%%%%%%%
%  Capítulo 1: Introducción- DEFINIR OBJETIVOS DE LA TESIS              %
%%%%%%%%%%%%%%%%%%%%%%%%%%%%%%%%%%%%%%%%%%%%%%%%%%%%%%%%%%%%%%%%%%%%%%%%%

\chapter{Generalidades}

\section{Presentación} % section headings are printed smaller than chapter names
El presente trabajo presenta los resultados obtenidos en desarrollo de la práctica empresarial, donde se modelaron dos procesos seleccionados a de los trabajados en la práctica y se modelaron bajo el estándar de BPMN, también se automatizaron distintas actividades a distintos colaboradores del banco de Bogotá haciendo uso de herramientas de Microsoft, se busca comparar los tiempos de ejecución de distintas actividades o procesos que hacían los colaboradores del banco y comparar estos para validar si la automatización optimiza el tiempo de las personas a las que se les desarrolló una solución.
%%%%%%%%%%%%%%%%%%%%%%%%%%%%%%%%%%%%%%%%%%%%%%%%%%%%%%%%%%%%%%%%%%%%%%%%%
%                           Objetivo                                    %
%%%%%%%%%%%%%%%%%%%%%%%%%%%%%%%%%%%%%%%%%%%%%%%%%%%%%%%%%%%%%%%%%%%%%%%%%

\section{Objetivos}
\subsection{Objetivo General}
Automatizar procesos y digitalizar actividades de las diferentes áreas del banco de Bogotá
solicitadas por los colaboradores del banco mediante mentorías empleando las herramientas
de Microsoft 365 y Microsoft Power Platform.

\subsection{Objetivos específicos}

\begin{itemize}
	\item Analizar y modelar procesos asignados por los colaboradores del banco de Bogotá.
	\item Automatizar los procesos requeridos por los colaboradores  del banco mediante las     herramientas de Microsoft 365 y Power Platform. 
	\item Validar el desempeño de los procesos automatizados para poder comparar los
	tiempos de ejecución de la actividad o tarea antes y después de la automatización del proceso.
\end{itemize} 


%%%%%%%%%%%%%%%%%%%%%%%%%%%%%%%%%%%%%%%%%%%%%%%%%%%%%%%%%%%%%%%%%%%%%%%%%
%                   Planteamiento del problema                          %
%%%%%%%%%%%%%%%%%%%%%%%%%%%%%%%%%%%%%%%%%%%%%%%%%%%%%%%%%%%%%%%%%%%%%%%%%

\section{Planteamiento del problema}
El banco de Bogotá es una entidad en Colombia, su sede principal se encuentra ubicada en Bogotá, es un banco líder en el mercado de empresas, personas y sector social. Tiene como objetivo ser un banco siempre a la vanguardia para brindar a sus clientes soluciones anticipadas, que les permitan vivir una experiencia bancaria satisfactoria. Además, hace parte del holding financiero más grande del país y uno de los mayores grupos bancarios de Latinoamérica.

Dentro del banco se maneja información de los clientes como nombres, apellidos, documento, etc. Hasta información bancaria e incluso la propia información de los colaboradores que trabajan con el banco, la información que fluye dentro de la organización llega a ser muy elevada, ya que por cada colaborador o cliente debe recolectar, almacenar y distribuir toda esa información dentro de la organización y debe estar disponible siempre que se necesite. El banco se apoya en múltiples herramientas de software para recolectar, almacenar o distribuir esta información como Microsoft Office herramienta ofimática, SAP software para la gestión de los procesos de negocios o CRM Poeplesoft software que permite desarrollar aplicaciones.

Pese al uso de estas herramientas, hay flujos de información o procesos que tardan demasiado o no se realiza de la manera adecuada, procesos que no tienen una forma definida de como recolectar la información o como o donde guardarla para poder procesarla y distribuirla según sea el caso. Estos procesos pueden traen problemas o errores en los flujos de información, como pueden ser la dispersión de datos al no tener centralizada la información. Para corregir los errores de estos procesos y la información pueda fluir de la mejor manera los colaboradores tienen que solventar dichos errores, la corrección de estos errores puede incurrir en actividades muy extensas o ser muy repetitivos, esto le quita tiempo al colaborador que puede emplear en la realización de otras actividades. Para esto el banco de Bogotá adquirió una serie de licencias de Microsoft llamadas Microsoft 365 y Microsoft Power Platform, esta última licencia contiene herramienta como Power Automate una herramienta o programa con la cual se pueden hacer estos flujos de trabajo y automatizar estas actividades para que los colaboradores no dediquen tanto tiempo en todos estos procesos como se ha demostrado en el caso de Toyota \citep{Microsoft2017}, donde los empleados de las instalaciones también han
desarrollado su propia aplicación para impulsar la eficiencia y mejorar la seguridad en los campus de gran tamaño de Toyota o
Virgin Atlantic donde se desarrolló una aplicación para su equipo de Clubhouse Spa que ayuda a los empleados a realizar un
seguimiento de los historiales de los pasajeros y asegurarse de que hayan completado los cuestionarios de salud necesarios
\citep{Atlantic2019}, tenemos también Power Apps que es una herramienta low code o herramienta de poco código en la cual se pueden crear aplicaciones personalizadas que optimizan los procesos y pueden mejorar la productividad, Power BI que es una herramienta para mostrar cantidades grandes de información de una forma agradable y facilita la comprensión de estos datos, SharePoint es una herramienta en la cual podemos crear sitios para compartir información o usarlos como una biblioteca de archivos, entre otras herramientas.

%%%%%%%%%%%%%%%%%%%%%%%%%%%%%%%%%%%%%%%%%%%%%%%%%%%%%%%%%%%%%%%%%%%%%%%%%
%                           Metodología                                 %
%%%%%%%%%%%%%%%%%%%%%%%%%%%%%%%%%%%%%%%%%%%%%%%%%%%%%%%%%%%%%%%%%%%%%%%%%
\section{Metodología}


\subsection[Enfoque]{Enfoque}
Para la realización del presente trabajo de grado, que tiene como objetivo mostrar como podemos optimizar el tiempo que se emplea en actividades por medio de la automatización  o digitalización de actividades, el trabajo se realizó tiene un enfoque cuantitativo.

\subsection[Alcance]{Alcance}
El banco de Bogotá en su estructura general se divide y subdivide en vicepresidencias y estás a su vez en direcciones, al momento de realizar la práctica me encuentro en las vicepresidencias de tecnología en un equipo llamado Smart Digital Workspace el cual es el encargado de realizar automatizaciones que llegan por solicitudes de las diferentes vicepresidencias del banco, las solicitudes que llegan al equipo son asignadas por un líder el cual se encarga de distribuir dichas solicitudes entre los desarrolladores que hacen parte del equipo.

\begin{itemize}
	\item Se elaboró un modelado de dos procesos con el estandar BPMN.
	\item Se realizó encuestas a los colaboradores a los cuales se les desarrolle una solución.
\end{itemize}


\subsection{Fuentes de información}
Como de fuentes de información para este proyecto se tomaron artículos relacionados con el tema o trabajos realizados anteriormente donde se estudie la automatización de procesos o contengan información sobre el estándar utilizado para el modelado de procesos que se utilizó en el proyecto.

\subsection[Población y muestra]{Población y muestra}

El banco de Bogotá tiene cuenta con alto número de colaboradores, los cuales podemos considerar como muestra del proyecto, pero dada la cantidad de colaboradores que tiene el banco y que no todos hacen solicitudes de automatización al equipo se tomará como muestra a aquellas personas que hagan solicitudes de automatización al equipo del cual hago parte.

Para determinar la muestra no se sigue alguna fórmula matemática, por lo que se tomará la muestra a conveniencia e irán relacionadas con los dueños de los procesos que se seleccionaron para modelar.

\subsection{Instrumentos}
Los principales instrumentos o herramientas a utilizar son una aplicación web gratis llamada BPMN.io una herramienta online y gratis, la cual se utilizó para modelar los procesos seleccionados, Microsoft Power Automate, Microsoft Power Apps, Microsoft Forms, Microsoft SharePoint estas herramientas de Microsoft serán las utilizadas para la automatización o digitalización de procesos durante la práctica estas herramientas se utilizan bajo la licencia que tiene el banco. Como instrumento de validación de las soluciones desarrolladas se valida y aplica una encuesta.


%%%%%%%%%%%%%%%%%%%%%%%%%%%%%%%%%%%%%%%%%%%%%%%%%%%%%%%%%%%%%%%%%%%%%%%%%
%                         Contribuciones                                %
%%%%%%%%%%%%%%%%%%%%%%%%%%%%%%%%%%%%%%%%%%%%%%%%%%%%%%%%%%%%%%%%%%%%%%%%%

%%%%%%%%%%%%%%%%%%%%%%%%%%%%%%%%%%%%%%%%%%%%%%%%%%%%%%%%%%%%%%%%%%%%%%%%%
%                           Estructura de la tesis                      %
%%%%%%%%%%%%%%%%%%%%%%%%%%%%%%%%%%%%%%%%%%%%%%%%%%%%%%%%%%%%%%%%%%%%%%%%%

\section{Estructura del documento}

Este documento está organizado en seis capítulos, en el primero se presentan las generalidades del proyecto, incluyendo los objetivos, la definición del problema y la metodología de práctica realizada.
El segundo capítulo describe los principales conceptos que estructuran el objeto de estudio y presenta un breve estado del arte relacionado con los principales trabajos de investigación que abordan de una u otra manera el objeto de estudio.
El tercer y cuarto capítulo presenta el modelado propuesto en el estándar BPMN y su implementación, quinto capítulo presenta la construcción, validación y los resultados obtenidos del artefacto de medición (Formulario) y el sexto capítulo nos presenta las conclusiones del proyecto.