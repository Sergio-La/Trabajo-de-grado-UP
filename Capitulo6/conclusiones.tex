\chapter{Conclusiones}

Como se pueden observar en los resultados de las encuestas (\ref{fig:vice} a \ref{fig:satis}) las automatizaciones, además de evitar el trabajo manual por parte de los colaboradores, se optimizó la ejecución en un 100\% para todos los encuestados (\ref{fig:optimizo}) y que el tiempo de ejecución del proceso (\ref{fig:antes} y \ref{fig:despues}) disminuyó considerablemente, todo esto en caso de los flujos en Power Automate, con aplicaciones desarrolladas en Power apps se evidencia que la aplicación optimizó el tiempo de trabajo (\ref{fig:tiempodetra}), y donde el 88\% estás satisfecho con la solución desarrollada.
\newline
El estandar BPMN es un estándar muy completo, fácil de aprender y fácil de usar, nos permite modelar procesos de negocio para entender como fluye la información entre los distintos participantes al desglosar los procesos en participantes, eventos, actividades, etc. El modelado de los procesos nos permite diseñar las automatizaciones de manera adecuada, eso porque el modelo nos representa como viaja la información a que participantes, donde sé almacena, cuando inicia, cuando finaliza, lo que hace que la construcción de los flujos y las aplicaciones sea más fácil.
\newline
Basándonos en los resultados de las encuestas, podemos concluir que las automatizaciones desarrolladas por medio de Power automate pueden automatizar la totalidad de un proceso o un porcentaje muy alto de este, mejora considerablemente el tiempo de ejecución de los procesos donde procesos de una hora podían realizarse en dos minutos con una automatización.
\newline
Las aplicaciones desarrolladas en Power automate optimizan el tiempo de trabajo en un 100\% según los resultados de las encuestas, además que son fáciles de utilizar, muestra los mensajes de errores necesarios y en general cumple la función de optimizar el trabajo de los colaboradores que la utilizan.