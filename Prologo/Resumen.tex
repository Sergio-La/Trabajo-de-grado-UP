\begin{prologo}{Resumen}      
	El Banco de Bogotá, es una empresa con miles de colaboradores (41,000) y millones de clientes activos (24 millones) (Concesión La Pintada 2017), por lo tanto, el flujo de información que se manipula es muy grande, por esta razón la entidad se apoya en herramientas de software como Microsoft Office, SAP, CRM Peoplesof, entre otros. A pesar de que ya se tiene la mayoría de los procesos digitalizados, en algunos procesos existen problemas en la recolección de datos, almacenamiento, en muchos de estos procesos se presentan repetición de actividades o las actividades para solventar estos problemas son muy extensas o manuales, lo cual hace que los colaboradores inviertan mucho tiempo en ellas. \\
	
	Como ejemplo, existen algunos  procesos en los cuales se le debe notificar a un colaborador de algún evento, estas notificaciones llegan de fuentes distintas, esto es un problema porque alguna notificación se puede perder y no atender lo solicitado, lo que generaría pérdidas en el banco. También existen procesos donde la información no se encuentra centralizada debido a que la persona o personas encargadas del proceso no establecen un único medio o herramienta para almacenar la información, este también es un problema que existe en muchos de ellos. El fin de automatizar una actividad es ahorrarle tiempo al colaborador y por consiguiente recursos al banco, ya que el tiempo ahorrado por este se puede emplear en otras actividades y así llegar a ser más productivo. Para llegar a ese fin, en este proyecto se emplearon las diferentes herramientas que ofrece Microsoft Power 356 y Microsoft Power Platform las cuales permitieron la digitalización de actividades y en otros casos la automatización para mejora de los procesos del banco.
\end{prologo}



